@ARTICLE{Keerti2022-nc,
  title    = "{DNA} fingerprinting: Use of autosomal short tandem repeats in
              forensic {DNA} typing",
  author   = "Keerti, Akshunna and Ninave, Sudhir",
  abstract = "Short tandem repeat (STR) markers for autosomal STR are used in
              forensic deoxyribonucleic acid (DNA) typing to track down the
              missing, verify family connections, and potentially connect
              suspects to crime sites. It is well acknowledged that
              forensically relevant genetic markers cannot predict phenotype.
              There is no evidence to support the claim that directly using
              forensic STR variations causes or indicates illness. Such an
              example would have significant ethical and permissible
              repercussions. It is essential to check the necessity to alert a
              blood donor or if a medical problem is identified during routine
              sample analysis. In this study, we assess the likelihood that
              forensic STRs might offer details beyond those required for
              primary identification. However, as the role of non-coding STRs
              in gene regulation is understood, the probability of discovering
              meaningful links is rising. For this review, Google Scholar,
              ScienceDirect, PubMed, and Google Search were all used to conduct
              a thorough electronic literature search. If they linked to the
              topic, thoughts, retrospective studies, observational studies,
              and first publications were considered. The case studies
              presented here highlight the critical role forensic DNA typing
              plays in reducing criminal risk and delivering conclusive
              evidence in cases. The primary method for forensic DNA typing is
              short tandem repeat (STR) typing. This discussion on the
              importance of STR markers to the criminal justice system is part
              of the present study. As unflinching proof of false beliefs and
              invaluable connections to the genuine culprits, DNA typing offers
              proof that may be utilised to prosecute and punish criminals. It
              may even deter certain offenders from committing more terrible
              offences. Additionally, forensic experts have used DNA typing
              techniques to re-examine ancient cases previously closed due to a
              lack of evidence.",
  journal  = "Cureus",
  volume   =  14,
  number   =  10,
  pages    = "e30210",
  month    =  oct,
  year     =  2022,
  keywords = "dna database; dna probe; forensic dna typing; restriction
              fragment length polymorphism; str typing",
  language = "en"
}


@inproceedings{fisch2014physical,
  title={Physical zero-knowledge proofs of physical properties},
  author={Fisch, Ben and Freund, Daniel and Naor, Moni},
  booktitle={Advances in Cryptology--CRYPTO 2014: 34th Annual Cryptology Conference, Santa Barbara, CA, USA, August 17-21, 2014, Proceedings, Part II 34},
  pages={313--336},
  year={2014},
  organization={Springer}
}